\chapter{Introducción y Justificación}

La segmentación precisa y automática de tumores cerebrales es crucial
para la planeación quirúrgica, la radioterapia y el seguimiento de la
enfermedad. Las redes neuronales convolucionales, y en particular la
arquitectura \textit{U-Net} \cite{ronneberger2015unet}, han mostrado un
desempeño sobresaliente.

\section*{Objetivo general}
Analizar y documentar el proceso de segmentación de tumores cerebrales
con U-Net, contrastando distintas implementaciones y buenas prácticas.

\section*{Objetivos específicos}
\begin{enumerate}[label=\alph*)]
  \item Implementar una U-Net 2-D/3-D básica y evaluar su rendimiento.
  \item Replicar el ejemplo 3-D U-Net de MATLAB y comparar resultados.
  \item Estudiar las modificaciones y la pérdida Dice multiclase de \cite{isensee2018brats}.
\end{enumerate}
